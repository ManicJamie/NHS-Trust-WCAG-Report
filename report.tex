\documentclass[12pt]{third-rep}

%% Any characters from a % to the end of line are comments.

%% The third-rep class and this starter kit were written by 
%% Graham Gough <graham@cs.man.ac.uk>
%% If you have any comments or questions regarding this document,
%% please post them to the local newsgroup man.cs.tex.

%% This skeleton report is organised as a master file called
%% report.tex which then includes files for individual parts including
%% abstract.tex, chapter1.tex, chapter2.tex, chapter3.tex and
%% appendix1.tex.  

%% The third-rep style is a locally created style based on the
%% standard LaTeX report style. If you really want to have a look at
%% it, its source can be found in
%% /usr/local/share/texmf/tex/latex/mancs/third-rep.cls
%%
%% More information about LaTeX in general and the local setup in
%% particular can be found on the web at 
%% http://csis.cs.manchester.ac.uk/software/contrib/latex
%%
%%%%%%%%%%%%%%%%%%%%%%%%%%%%%%%%%%%%%%%%%%%%%%%%%%%%%%%%%%%%%%%%%%%%%%%%
%%
%% This is an example of how you load extra packages.
%% Some packages are already loaded in the third-rep class

\usepackage{url} % typeset URL's sensibly

\usepackage{pslatex} % Use Postscript fonts

%% The best way to latex just one chapter is to uncomment lines such as
%% the next:
%\includeonly{chapter1}

%% This defines the title (the \\ forces a line break)
\title{Assessing Accessibility of NHS Trust Websites}
%% and author
\author{J. Bloomfield}
%% and supervisor
\supervisor{Dr.\ A. Lecturer}
%% and the year of the report
\reportyear{2024}

%% this defines the file that contains the text of the abstract, there
%% must be one of these by the time you submit your report.
\abstractfile{abstract.tex}

%% this defines the file that contains the acknowledgements (it can be
%% omitted if you don't feel like thanking anyone)
\thanksfile{thanks.tex}

%% Uncomment the following lines if you want to include the date as a
%% header in draft versions. See the documentation for fancyhdr for
%% more ways of modifying headers (texdoc fancyhdr will show you the
%% docs) 

\usepackage{fancyhdr}
\pagestyle{fancy}
\lhead{}  % left head
\chead{Draft: \today} % centre head
\lfoot{}
\cfoot{\thepage}
\rfoot{}

%% The following line sets up the use of PostScript fonts rather
%% than the standard bitmapped fonts.
\usepackage{pslatex}

%% Uncomment the following line if you want to change the name of the
%% Bibliography to References
%\renewcommand{\bibname}{References}

\usepackage{listings}

%% End of preamble, the actual document starts here
%%

\begin{document}

\setlength{\headheight}{15pt}

%% This actually creates the title and abstract pages
\dotitleandabstract\

%% Generate contents etc
\tableofcontents
\listoffigures
\listoftables

%% These include the actual text
\chapter{Introduction}\label{cha:intro}

%TODO: split context into several pieces

\section{Context and Motivation}
The advent of the Internet has catalysed an unprecedented increase in the availability of information, from business open hours to academic papers. Included in this shift is the UK's National Health Service; many parts of a patient's care can now be accessed via an NHS website, from contact details to booking test appointments. This generally represents a massive improvement in patients' access to care where they can access the services they need from a website.

Web pages are a purely visual and predominantly text-based medium, which presents problems for visually impaired users, as well as users with language processing or learning disabilities. Many disabled users use accessibility software, most often screen readers in tandem with keyboard navigation, to allow them to navigate the web. However, if a website is not designed with screen readers in mind, disabled users may be unable to access information or services on the web page. In the case of NHS Trusts, this means disabled users may be unable to book appointments or access details relevant to their secondary care.

\section{Objectives}\label{sec:objectives}
To better organise the work to be done on this project, the objectives have been split into the following sub-tasks:
\begin{itemize}
    \item Research the existing tooling for automatic accessibility testing, as well as their respective coverage of Web Content Accessibility Guidelines 2.1 and 2.2
    \item Develop a method of automatically capturing snapshots of NHS trust websites, either complete or partial
    \item Save a history of accessibility outcomes to allow statistical tracking over time
    \item Create a website to present reports in a user-friendly and insightful manner
    \item Perform statistical analysis on reports to shed insights on potential contributors to websites' accessibility or lack thereof
\end{itemize}

\section{Outcomes} %% Place project outcomes here

\section{Report Structure}
\begin{itemize}
    \item 
    \item 
    \item 
    \item 
\end{itemize}
\chapter{Background}\label{cha:background}

% expand on what others have done

\section{Web Content Accessibility Guidelines}

\section{Gaps in knowledge}
\chapter{Methodology}\label{cha:methodology}
\chapter{Results}\label{cha:results}

% Artefact description & evidence

\section{Completed Artefact}

\subsection{Report generator}

\subsection{WCAG Report Website}

\section{Data analysis}

\section{Comparison to baseline}
\chapter{Reflection}\label{cha:reflection}
\chapter{Discussion}\label{cha:discussion}
\chapter{Conclusion}\label{cha:conclusion}

~\cite{boyle} % placeholder to avoid BibTeX compile failure

\bibliography{refs}             % this causes the references to be
                                % listed

\bibliographystyle{alpha}       % this determines the style in which
                                % the references are printed, other
                                % possible values are plain and abbrv
%% Appendices start here
\appendix
\chapter{Example of operation}

An appendix is just like any other chapter, except that it comes after
the appendix command in the master file.

One use of an appendix is to include an example of input to the system
and the corresponding output.

One way to do this is to include, unformatted, an existing input file. 
You can do this using \verb=\verbatiminput=. In this appendix we
include a copy of the C file \textsf{hello.c} and its output file
\textsf{hello.out}. If you use this facility you should make sure that
the file which you input does not contain \texttt{TAB} characters,
since \LaTeX\ treats each \texttt{TAB} as a single space; you can use
the Unix command \texttt{expand} (see manual page) to expand tabs into
the appropriate number of spaces. 

\section{Example input and output}\label{sec:inp-eg}
\subsection{Input}\label{sec:input}
(Actually, this isn't input, it's the source code, but it will do as
an example)

\verbatiminput{hello.c}

\subsection{Output}\label{sec:output}

\verbatiminput{hello.out}
\subsection{Another way to include code}
You can also use the capabilities of the \texttt{listings} package to
include sections of code, it does some keyword highlighting.

\lstinputlisting[language=C]{hello.c}

% Local Variables: 
% mode: latex
% TeX-master: "report"
% End: 

\end{document}
