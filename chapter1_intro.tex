\chapter{Introduction}\label{cha:intro}

%TODO: split context into several pieces

\section{Context and Motivation}
The advent of the Internet has catalysed an unprecedented increase in the availability of information, from business open hours to academic papers. Included in this shift is the UK's National Health Service; many parts of a patient's care can now be accessed via an NHS website, from contact details to booking test appointments. This generally represents a massive improvement in patients' access to care where they can access the services they need from a website.

Web pages are a purely visual and predominantly text-based medium, which presents problems for visually impaired users, as well as users with language processing or learning disabilities. Many disabled users use accessibility software, most often screen readers in tandem with keyboard navigation, to allow them to navigate the web. However, if a website is not designed with screen readers in mind, disabled users may be unable to access information or services on the web page. In the case of NHS Trusts, this means disabled users may be unable to book appointments or access details relevant to their secondary care.

\section{Objectives}\label{sec:objectives}
To better organise the work to be done on this project, the objectives have been split into the following sub-tasks:
\begin{itemize}
    \item Research the existing tooling for automatic accessibility testing, as well as their respective coverage of Web Content Accessibility Guidelines 2.1 and 2.2
    \item Develop a method of automatically capturing snapshots of NHS trust websites, either complete or partial
    \item Save a history of accessibility outcomes to allow statistical tracking over time
    \item Create a website to present reports in a user-friendly and insightful manner
    \item Perform statistical analysis on reports to shed insights on potential contributors to websites' accessibility or lack thereof
\end{itemize}

\section{Outcomes} %% Place project outcomes here

\section{Report Structure}
\begin{itemize}
    \item 
    \item 
    \item 
    \item 
\end{itemize}